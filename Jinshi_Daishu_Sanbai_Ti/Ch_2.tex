\documentclass[11pt]{article}
\usepackage{amsmath}
\usepackage{amsfonts}
\usepackage{amssymb}
\usepackage{amsthm}
\usepackage{graphicx}

\newcommand{\dd}{\mathrm{d}}
\newcommand{\pd}{\partial}
\newcommand{\mr}{\mathbb{R}}

\newtheorem{problem}{Problem}
\numberwithin{problem}{section}
\title{Chapter 2. Ring Theory}
\author{Hu Zheng \\ Department of Mathematics, Zhejiang University}
\newenvironment{solution}
               {\let\oldqedsymbol=\qedsymbol
                \renewcommand{\qedsymbol}{$\blacktriangleleft$}
                \begin{proof}[\bfseries\upshape Solution:]}
               {\end{proof}
                \renewcommand{\qedsymbol}{\oldqedsymbol}}

\begin{document}

\maketitle

\section{Basic Notions}

\begin{problem}[Kaplansky]
If some element in a unital ring has more than one right inverse, then it has infinitely many inverses.
\begin{solution}

\end{solution}
\end{problem}

\begin{problem}
\begin{itemize}


\item[(1)] Suppose $L$ is a noncommutative field, and $a$ is out of the center of $L$, then $L$ is generated by all the congruent elements of $a$.
\item[(2)] Suppose $L$ is a field and $K$ its proper subfield, and $K^*=K-\{0\}$ is a normal subgroup of $L^*$, then $K$ is contained in the center of $L$.
\end{itemize}
\begin{solution}

\end{solution}

\end{problem}


\end{document}
